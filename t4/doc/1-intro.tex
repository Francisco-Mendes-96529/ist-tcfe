\section{Introduction}
\label{sec:introduction}

% state the learning objective 
The purpose of this laboratory assignment is to make an Audio Amplifier, with an $8\Omega$ speaker, with the goal to get the highest merit, M, possible.
$$ M = \frac{voltageGain \times bandwidth}{cost \times lowerCutoffFreq}$$

Using the circuit shown in Figure~\ref{fig:t4-audioamplifier}, we tested different values for the resistors and capacitors and we found that the values in Table~\ref{tab:values} yielded the best merit.

In Section~\ref{sec:simulation}, the circuit is analyzed by simulation using the software Ngspice. 

In Section~\ref{sec:analysis}, an approximated theoretical model, using the software GNU Octave, of the circuit in Figure~\ref{} is presented. 

In Section~\ref{sec:comparison}, a comparison is done between the results obtained by both analyses, theoretical and simulation.

The conclusions of this study are outlined in Section~\ref{sec:conclusion}.

\begin{table}[ht!]
    \centering
    \begin{tabular}{c c}
    \toprule
    Component & Value ($k\Omega$ or $mF$) \\ \midrule
    $R_{in}$  & 0.1 \\
    $C_i$     & 0.2 \\
    $R_1$     & 80  \\
    $R_2$     & 20  \\
    $R_C$     & 1   \\
    $R_E$     & 0.1 \\
    $C_B$     & 2   \\
    $R_{out}$ & 0.1 \\
    $C_O$     & 2   \\
    \end{tabular}
    \label{tab:values}
    \caption{Resistance and Capacitance for the components}
\end{table}

\begin{comment}
\begin{figure}[ht!]
\centering
    \includegraphics[width=0.8\linewidth]{t3-transformer.pdf}
\caption{Circuit T3, with transformer.}
\label{fig:t3-transformer}
\end{figure}

\begin{figure}[ht!]
\centering
    \includegraphics[width=0.8\linewidth]{t3-depSources.pdf}
\caption{Circuit T3, with transformer equivalent.}
\label{fig:t3-depSources}
\end{figure}

\begin{figure}[ht!]
\centering
    \includegraphics[width=0.8\linewidth]{t3-lp.pdf}
\caption{Circuit T3, with low-pass filter.}
\label{fig:t3-lp}
\end{figure}
\end{comment}
\FloatBarrier
\clearpage
