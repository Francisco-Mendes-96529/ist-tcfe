\section{Theoretical Analysis}
\label{sec:analysis}

In this section, the circuit shown in Figure~\ref{fig:t3-depSources} is analyzed
theoretically, according to the following steps.


\subsection{Step 1}\label{subsec:step1}
The circuit was analyzed considering the voltage across C as constant,
$$\dot{v}_C = 0 \Rightarrow i_2=i_D$$,
resulting in the following equations:

\begin{equation}\label{step1}
  \begin{split}
    v_2 &= v_D + v_R + 25v_D + v_D\\
    &= 27v_D + RI_s(e^{\frac{v_D}{V_T}}-1)\\
  \end{split}
\end{equation}
\vspace{-15pt}
\begin{align}
  f(v_D)&=27v_D + RI_s(e^{\frac{v_D}{V_T}}-1) - v_2\\
  f'(v_D)&=27 + \frac{RI_s}{V_T}e^{\frac{v_D}{V_T}}\\
  v_C&=RI_S(e^{\frac{v_D}{V_T}}-1) + 25v_D
\end{align}




\subsection{Step 2}\label{subsec:step2}
The circuit was analyzed considering the voltage $v_2 < v_C$ and therefore the capacitor is
discharging, resulting in the following equations:

\begin{align}\label{step2}
  f(v_D)&=25v_D + RI_s(e^{\frac{v_D}{V_T}}-1) - v_C\\
  f'(v_D)&=25 + \frac{RI_s}{V_T}e^{\frac{v_D}{V_T}}
\end{align}
\vspace{-15pt}
\begin{equation}
  \begin{split}
    i_C &= C\dot{v}_C\\
    &=-i_D\\
    &=-I_s(e^{\frac{v_D}{V_T}}-1)
  \end{split}
\end{equation}
\vspace{-15pt}
\begin{equation}
\begin{split}
  \dot{v}_C &= -\frac{I_s}{C}(e^{\frac{v_D}{V_T}}-1)\\
  \Leftrightarrow v_{C\ next} &= -\frac{I_s}{C}(e^{\frac{v_D}{V_T}}-1)*h + v_C
\end{split}
\end{equation}



\subsection{Results}
The results obtained are shown in Table~\ref{tab:vout}, and the plots in
Figure~\ref{fig:dc-23-45}.\\

\begin{minipage}[b]{0.48\textwidth}
  \centering
  \input{../mat/tabvout}
  \captionsetup{type=table}
  \caption{Results obtained.}
  \label{tab:vout}
\end{minipage}
\hfil
\begin{minipage}[b]{0.48\textwidth}
  \centering
  \includegraphics[width=0.9\textwidth]{vT3.pdf}
  \captionsetup{type=figure}
  \caption{Plots obtained.}
  \label{fig:dc-23-45}
\end{minipage}
