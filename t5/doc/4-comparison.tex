\section{Comparison}
\label{sec:comparison}

Comparing the results achieved in the simulation and in the theoretical analysis, Table~\ref{tab:comp} and Figures~\ref{fig:votf} and \ref{fig:gain}, it is possible to see some differences: gain and phase differ for frequencies greater than the central frequency.

This discrepancy may have resulted from assuming an ideal op-amp model in the theoretical analysis, as shown from the behaviour for high frequencies.\\


\begin{table}[ht!]
  \centering
    \begin{tabular}{|c|c|c|}
      \hline    
       & \textbf{Theoretical} & \textbf{Simulation} \\ \hline
      \input{../mat/tab-line1} & \input{../sim/freq1_tab}
      \input{../mat/tab-line2} & \input{../sim/freq2_tab}
      \input{../mat/tab-line3} & \input{../sim/freq3_tab}
      \input{../mat/tab-line4} & \input{../sim/zin_tab}
      \input{../mat/tab-line5} & \input{../sim/zout_tab}
      \input{../mat/tab-line6} & \input{../sim/gain1_tab}
      \input{../mat/tab-line7} & \input{../sim/gain2_tab}
    \end{tabular}
\caption{Comparison between theoretical and simulation values}
\label{tab:comp}
\end{table}


\subsection{Practical results}

In the laboratory, it was possible to simulate with real components the circuit in Figure~\ref{fig:t5-real}. For frequency $f = 1\ kHz$, the results obtained were:

\begin{align*}
  v_i &= 170\ mV \\
  v_o &= 16,7\ V \\
  Gain &\approx 98,235 \\
  Gain_{dB} &\approx 39,85 \\
  Gain\ deviation &\approx 0,32 \%
\end{align*}

The results are according to the the simulation analysis. These differences may occur because of the resistors and capacitors tolerance, and the resistance of the wires.

\clearpage
