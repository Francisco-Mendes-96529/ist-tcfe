\section{Theoretical Analysis}
\label{sec:analysis}

In this section, the circuit shown in Figure~\ref{fig:t2} is analysed
theoretically, according to the steps mentioned in Section~\ref{sec:introduction}.

\subsection{Step 1}
The circuit was analyzed for $t<0$.

Resulting in the following matrix system:


\begin{equation}\label{meshM}
  \begin{bmatrix}
    1 & 0 & 0 & 0 & 0 & 0 & 0 \\
    G_1 & -G_1-G_2-G_3 & G_2 & G_3 & 0 & 0 & 0 \\
    0 & K_b+G_2 & -G_2 & -K_b & 0 & 0 & 0 \\
    0 & G_3 & 0 & -G_3-G_4-G_5 & G_5 & G_7 & -G_7 \\
    0 & -K_b & 0 & G_5+K_b & -G_5 & 0 & 0 \\
    0 & 0 & 0 & 0 & 0 & -G_6-G_7 & G_7 \\
    0 & 0 & 0 & 1 & 0 & K_dG_6 & -1 \\
  \end{bmatrix}
  \begin{bmatrix}
    V_1\\
    V_2\\
    V_3\\
    V_5\\
    V_6\\
    V_7\\
    V_8\\
  \end{bmatrix}
  =
  \begin{bmatrix}
    V_s\\
    0\\
    0\\
    0\\
    0\\
    0\\
    0\\
  \end{bmatrix}
\end{equation}

Solving the system~\ref{meshM} results in the values in the Table~\ref{tab:teo1}.

\begin{table}[ht!]
  \centering
  \begin{tabular}{cc}
    %\input{../mat/nodalAn-tneg.tex}
  \end{tabular}
  \caption{Voltages in each node, for $t<0$.}
  \label{tab:teo1}
\end{table}







\subsection{Step 2}
The circuit was analyzed, for $t=0$, as shown in Figure~\ref{fig:t2-2}.


{\color{red}
\Huge EXPLAIN
}

\begin{figure}[ht!]
  \centering
  %\includegraphics[width=0.4\textwidth]{t2-2.pdf}
  \caption{Circuit, for $t=0$.}
  \label{fig:t2-2}
\end{figure}

Resulting in the following matrix system:

\begin{equation}\label{nodeM}
  \begin{bmatrix}
    -G_1-G_2-G_3 & G_2 & G_3 & 0 & 0 & 0 & 0 \\
    K_b+G_2 & -G_2 & -K_b & 0 & 0 & 0 & 0 \\
    G_3 & 0 & -G_3-G_4-G_5 & G_5 & G_7 & -G_7 & -1 \\
    -K_b & 0 & G_5+K_b & -G_5 & 0 & 0 & 1 \\
    0 & 0 & 0 & 0 & -G_6-G_7 & G_7 & 0\\
    0 & 0 & 1 & 0 & K_dG_6 & -1 & 0\\
    0 & 0 & 0 & 1 & 0 & -1 & 0\\
  \end{bmatrix}
  \begin{bmatrix}
    V_2\\
    V_3\\
    V_5\\
    V_6\\
    V_7\\
    V_8\\
    I_x\\
  \end{bmatrix}
  =
  \begin{bmatrix}
    0\\
    0\\
    0\\
    0\\
    0\\
    0\\
    V_x\\
  \end{bmatrix}
\end{equation}

Solving the system~\ref{nodeM} results in the values in the Table~\ref{tab:teo2}.

\begin{table}[ht!]
  \centering
  \begin{tabular}{cc}
    %\input{../mat/nodalAn-t0.tex}
  \end{tabular}
  \caption{Voltage in each node, for $t=0$.}
  \label{tab:teo2}
\end{table}
\FloatBarrier



\subsection{Step 3}
The circuit was analyzed to determine the natural response in node 6, $v_{6n}(t)$, for $t\in [0,20]ms$.

{\color{red}
\Huge EXPLAIN
}
bla bla, Figure~\ref{fig:v6n}

\begin{figure}[ht!]
  \centering
  %\includegraphics[width=0.7\textwidth]{../mat/v6natural.eps}
  \caption{Natural response, $v_{6n}(t)$, for $t\in [0,20]ms$.}
  \label{fig:v6n}
\end{figure}
\FloatBarrier





\subsection{Step 4}
Analyze the circuit to calculate the phasors in each node, for $t\in [0,20]ms$ and $f=1kHz$.

{\color{red}
\Huge EXPLAIN
}
bla bla, Table~\ref{tab:teo4}


\begin{table}[ht!]
  \centering
  \begin{tabular}{cc}
    %\input{../mat/tabPhasors.tex}
  \end{tabular}
  \caption{Phasors in each node, for $t\in [0,20]ms$ and $f=1kHz$.}
  \label{tab:teo4}
\end{table}
\FloatBarrier





\subsection{Step 5}
The circuit was analyzed to determine the total solution (natural+forced) in node 6, $v_6(t)$, for $t\in [-5,20]ms$ and $f=1kHz$.

{\color{red}
\Huge EXPLAIN
}
bla bla, Figure~\ref{fig:v6}


\begin{figure}[ht!]
  \centering
  %\includegraphics[width=0.7\textwidth]{../mat/final.eps}
  \caption{Total solution, $v_{6}(t)$, for $t\in [-5,20]ms$ and $f=1kHz$.}
  \label{fig:v6}
\end{figure}
\FloatBarrier





\subsection{Step 6}
Determine the frequency responses $v_C(f),\ v_6(f)$, for the frequency range 0.1 Hz to 1 MHz.

{\color{red}
\Huge EXPLAIN
}
bla bla, Figures~\ref{fig:teo-db} and \ref{fig:teo-phase}.


\begin{minipage}[b]{0.48\textwidth}
\centering
    %\includegraphics[width=0.9\linewidth]{../mat/dB.eps}
    \captionsetup{type=figure}
\caption{Frequency logarithmic scale, magnitude in dB.}
\label{fig:teo-db}
\end{minipage}
\begin{minipage}[b]{0.48\textwidth}
\centering
    %\includegraphics[width=0.9\linewidth]{../mat/phase.eps}
    \captionsetup{type=figure}
  \caption{Frequency logarithmic scale, phase in degrees.}
  \label{fig:teo-phase}
\end{minipage}
