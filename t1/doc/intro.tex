\section{Introduction}
\label{sec:introduction}

% state the learning objective 
The objective of this laboratory assignment is to study a circuit containing an
independent voltage source $V_a$, an independent current source $I_d$, a current-controlled voltage source $V_c$, a voltage-controlled current source $I_b$, and 7 resistors. The circuit can be seen in Figure~\ref{fig:t1}. (Note: in the resistors, the current was considered to go through from the right side to the left side, and downwards.)


In Section~\ref{sec:analysis}, a theoretical analysis, using the software GNU Octave, of the circuit is
presented. In Section~\ref{sec:simulation}, the circuit is analysed by
simulation using the software Ngspice, and the results are compared to the theoretical results obtained in
Section~\ref{sec:analysis}. The conclusions of this study are outlined in
Section~\ref{sec:conclusion}.



After running the program ``t1\_datagen.py'', the following values were obtain in Table~\ref{tab:constants}.

\begin{minipage}[b]{0.48\textwidth}
\centering
    \includegraphics[width=0.9\linewidth]{t1.pdf}
    \captionsetup{type=figure}
\caption{Circuit T1.}
\label{fig:t1}
\end{minipage}
\begin{minipage}[b]{0.48\textwidth}
\centering
    \input{../mat/tabconstants.tex}
    \captionsetup{type=table}
  \caption{Components characteristics.}
  \label{tab:constants}
\end{minipage}
