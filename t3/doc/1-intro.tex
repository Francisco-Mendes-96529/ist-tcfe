\section{Introduction}
\label{sec:introduction}

% state the learning objective 
The purpose of this laboratory assignment is to make an AC/DC converter, from AC(230V / 50Hz) to DC(12V), with the goal to get the highest merit, M, possible.
$$ M = \frac{1}{cost*(ripple(v_O)+average(v_O-12)+10^{-6})}$$

After some simulations with different circuits, we found the circuit shown in Figure~\ref{fig:t3-transformer} yielded the best merit.

The circuit above is equivalent to the circuit in Figure~\ref{fig:t3-depSources}.

In Section~\ref{sec:simulation}, it is explained the process done and the circuit is analyzed by
simulation using the software Ngspice. 


In Section~\ref{sec:analysis}, an approximated theoretical model, using the software GNU Octave, of the circuit is presented. 

In Section~\ref{sec:comparison}, a comparison is done between the results obtained by both analyses, theoretical and simulation.

In Section~\ref{sec:extra},  a circuit with a low-pass filter, as shown in Figure~\ref{fig:t3-lp} is analyzed by
simulation using the software Ngspice. 

The conclusions of this study are outlined in
Section~\ref{sec:conclusion}.


\begin{figure}[ht!]
\centering
    \includegraphics[width=0.8\linewidth]{t3-transformer.pdf}
\caption{Circuit T3, with transformer.}
\label{fig:t3-transformer}
\end{figure}

\begin{figure}[ht!]
\centering
    \includegraphics[width=0.8\linewidth]{t3-depSources.pdf}
\caption{Circuit T3, with transformer equivalent.}
\label{fig:t3-depSources}
\end{figure}

\begin{figure}[ht!]
\centering
    \includegraphics[width=0.8\linewidth]{t3-lp.pdf}
\caption{Circuit T3, with low-pass filter.}
\label{fig:t3-lp}
\end{figure}

\FloatBarrier
\clearpage
