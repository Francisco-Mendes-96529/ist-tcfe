\section{Introduction}
\label{sec:introduction}

% state the learning objective 
The purpose of this laboratory assignment is to study a circuit that can be observed in Figure~\ref{fig:t2} containing an independent voltage source $V_s$, a current-controlled voltage source $V_d$, a voltage-controlled current source $I_b$,  seven resistors, and one capacitor $C$. The voltage $V_s$ varies in time according to the expression~1. (Note: in the resistors, the current was considered to go through from the right side to the left side, and downwards).
\begin{gather}\label{eq:vs}
    v_s(t) = V_s\,u(-t) + \sin(2\pi f\,t)\,u(t) \\
    u(t) =
    \begin{cases}
    0,\quad t<0\\
    1,\quad t\geqslant 0
    \end{cases}
\end{gather}


In Section~\ref{sec:analysis}, a theoretical analysis, using the software GNU Octave, of the circuit is presented. The first step in the analysis was to determine all the voltages in the nodes and the currents in the branches, for $t<0$, using the nodal method. Secondly, the equivalent resistance seen by the capacitor, for $t=0$, and the time constant, $\tau$ were calculated. The next step was to determine the natural response in node 6, for $t\in [0,20]ms$. In step 4, the phasors of each node were calculated, using the nodal method and replacing $C$ with its impedance $Z_C$. The fifth step, was to determine the final solution (natural+forced responses), for $f=1kHz$ and the time interval $t\in [-5,20]ms$. Finally, the frequency responses $v_C$ and $v_6$ were computed for the frequency range 0.1 Hz to 1 MHz.

In Section~\ref{sec:simulation}, the circuit is analyzed by
simulation using the software Ngspice, and the results are compared to the theoretical results obtained in
Section~\ref{sec:analysis}. 

In Section~\ref{sec:comparison}, is done a more detailed comparison between the results obtained by both analyses, theoretical and simulation.

The conclusions of this study are outlined in
Section~\ref{sec:conclusion}.

After running the program ``t2\_datagen.py'', the following values were obtained in Table~\ref{tab:constants}.

\begin{minipage}[b]{0.48\textwidth}
\centering
    \includegraphics[width=0.9\linewidth]{t2.pdf}
    \captionsetup{type=figure}
\caption{Circuit T2.}
\label{fig:t2}
\end{minipage}
\begin{minipage}[b]{0.48\textwidth}
\centering
   \input{../mat/tabconstants.tex}
    \captionsetup{type=table}
  \caption{Components characteristics.}
  \label{tab:constants}
\end{minipage}
