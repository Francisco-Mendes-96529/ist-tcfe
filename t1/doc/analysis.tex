\section{Theoretical Analysis}
\label{sec:analysis}

In this section, the circuit shown in Figure~\ref{fig:t1} is analyzed
theoretically, using the mesh method and the nodal method.

\subsection{Mesh Analysis}
The circuit was analyzed using the 4 elemental meshes, as shown in Figure~\ref{fig:mesh}

\begin{figure}[ht!]
  \centering
  \includegraphics[width=0.4\textwidth]{t1-mesh.pdf}
  \caption{Currents through the elemental meshes.}
  \label{fig:mesh}
\end{figure}

Resulting in the following equations:

\begin{equation}\label{meshEq}
  \begin{cases}
    -R_3(-I_A + I_B) - R_1(-I_A) + V_a + R_4(I_A-I_C) = 0 \\
    I_B = I_b\quad (by\ observation) \\
    -V_c - R_4(-I_C + I_A) + R_6I_C+ R_7I_C = 0 \\
    I_D = I_d\quad (by\ observation) \\
    I_b = K_bV_b = K_bR_3(-I_A + I_B) \\
    V_C = K_CI_C \\
  \end{cases}
\end{equation}

Simplifying in the next matrix system:

\begin{equation}\label{meshM}
  \begin{bmatrix}
    R_3+R_1+R_4 & -R_3 & -R_4 & 0 \\
    K_bR_3 & 1-K_bR_3 & 0 & 0 \\
    -R_4 & 0 & -K_C + R_4+R_6+R_7 & 0 \\
    0 & 0 & 0 & 1\\
  \end{bmatrix}
  \begin{bmatrix}
    I_A\\
    I_B\\
    I_C\\
    I_D\\
  \end{bmatrix}
  =
  \begin{bmatrix}
    -V_a\\
    0\\
    0\\
    I_d\\
  \end{bmatrix}
\end{equation}

Solving the system~\ref{meshM} results in the values in the Table~\ref{tab:mesh}.

\begin{table}[ht!]
  \centering
  \input{../mat/tabmesh.tex}
  \caption{Currents of each mesh.}
  \label{tab:mesh}
\end{table}
\FloatBarrier






\subsection{Nodal Analysis}
The circuit was analyzed using 7 nodes and the GND node, as shown in Figure~\ref{fig:node}

\begin{figure}[ht!]
  \centering
  \includegraphics[width=0.4\textwidth]{t1-node.pdf}
  \caption{Voltage in each node.}
  \label{fig:node}
\end{figure}

Resulting in the following equations:

\begin{equation}\label{nodeEq}
  \begin{cases}
    KCL\ equations:\\
    node\ 2\to (V_1-V_2)G_1 - (V_2-V_3)G_2 - (V_2-V_4)G_3 = 0\\
    node\ 3\to (V_2-V_3)G_2 + I_b = 0\\
    node\ 4\to (V_2-V_4)G_3 + (-V_4)G_4 - (V_4-V_5)G_5 - I_{V_C} = 0\\
    node\ 5\to (V_4-V_5)G_5 + I_d - I_b = 0\\
    node\ 6\to (-V_6)G_6 - (V_6-V_7)G_7 = 0\\
    \\
    Additional\ equations:\\
    V_1 = V_a\\
    I_b = K_b(V_2-V_4)\\
    I_{V_C} = I_d - (V_6-V_7)G_7\\
  \end{cases}
\end{equation}

Simplifying in the next matrix system:

\begin{equation}\label{nodeM}
  \begin{bmatrix}
    1 & 0 & 0 & 0 & 0 & 0 & 0 \\
    G_1 & -G_1-G_2-G_3 & G_2 & G_3 & 0 & 0 & 0 \\
    0 & K_b+G_2 & -G_2 & -K_b & 0 & 0 & 0 \\
    0 & G_3 & 0 & -G_3-G_4-G_5 & G_5 & G_7 & -G_7 \\
    0 & -K_b & 0 & G_5+K_b & -G_5 & 0 & 0 \\
    0 & 0 & 0 & 0 & 0 & -G_6-G_7 & G_7 \\
    0 & 0 & 0 & 1 & 0 & K_cG_6 & -1 \\
  \end{bmatrix}
  \begin{bmatrix}
    V_1\\
    V_2\\
    V_3\\
    V_4\\
    V_5\\
    V_6\\
    V_7\\
  \end{bmatrix}
  =
  \begin{bmatrix}
    V_a\\
    0\\
    0\\
    I_d\\
    -I_d\\
    0\\
    0\\
  \end{bmatrix}
\end{equation}

Solving the system~\ref{nodeM} results in the values in the Table~\ref{tab:node}.

\begin{table}[ht!]
  \centering
  \input{../mat/tabnode.tex}
  \caption{Voltage in each node.}
  \label{tab:node}
\end{table}
\FloatBarrier



\subsection{Results of both methods}
The voltage drop and currents of each resistor were the same of both methods and are shown in the Table~\ref{tab:IeV}.

\begin{table}[ht!]
  \centering
  \input{../mat/tabIeV.tex}
  \caption{Voltage drop and Currents of each resistor.}
  \label{tab:IeV}
\end{table}
\FloatBarrier
